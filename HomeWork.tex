\documentclass{report}
\usepackage[table]{xcolor}% http://ctan.org/pkg/xcolor
\usepackage{xepersian}
\settextfont{XB Niloofar}
\usepackage{setspace}
\usepackage{multirow}

\begin{document}

\title{ تکلیف سری پنجم روش تحقیق}
\author{محمد مهدی سمیعی پاقلعه\\ 9331064 }
\date{}
\maketitle
\section*{مقدمه:}

\paragraph{} \hspace{0pt}
در ۱۰۰ سال گذشته، بشر با پیشرفت های سریع و گسترده‌ای در حوزه علوم روباتیک و هوش مصنوعی مواجه بوده است. این پیشرفت ها در زمینه روباتیک با ساخته شدن روبات های انسان نمایی همراه بوده که قادرند اعمال حرکتی و مکانیکی انسانی را به خوبی انجام دهند. در زمینه هوش مصنوعی نیز دانشمندان این حوزه توانسته اند عامل های مصنوعی طراحی و پیاده‌سازی کنند که قادر به انجام رفتارهای انسانی باشند. اما این پیاده سازی ها مطابق آزمون تورینگ 
\LTRfootnote{Turing Test}
که میزان هوشمندی بر اساس کیفیت تقلید عامل مصنوعی از رفتار های انسانی تعریف شده انجام گرفته است. سوالی که اکنون مطراح می شود این است که آیا بشر در نهایت قادر خواهد بود تا جنبه فکری و ذهنی خود را به صورت کامل در عامل های مصنوعی که خود آن ها را ساخته است؛ پیاده سازی کند؟ آیا می توان روزی روبات و عاملی ساخت که با قطعیت گفت که این عامل فرقی با نوع انسان نخواهد داشت؟ آیا عامل های مصنوعی قادرند هر مسأله را که نوع انسان قابلیت حل آن را دارد؛ حل کنند؟


در این تحقیق مسأله نهایی ما، امکان شبیه سازی کامل جنبه های فکری انسان است. از این رو به بررسی امکان تحقق دو پرسش زیر توسط ماشین های محاسباتی می پردازیم:
\begin{enumerate}
  \item  آیا ماشین های محاسباتی قادرند هر آنچه را که انسان قابلیت حل آن را دارد حل کنند؟
  \item  آیا ماشین های محاسباتی دارای آگاهی و هوشیاری نسبت به آن چه که انجام می دهند هستند؟
\end{enumerate}

\section*{مرور سوابق موضوع:}

\paragraph{} \hspace{0pt}
در زمینه موضوع دانشمندان حوزه علوم هوش مصنوعی، ریاضی، منطق و فیلسوف های متعددی وارد شدند و به اظهار نظر پرداخته اند. نظرات این متخصصان درباره مسأله مطرح شده به سه دسته زیر تقسیم می شود:
\begin{enumerate}
	\item   دسته اول مانند آلن نیول 
	\LTRfootnote{Allen Newell}
	و هربرت سیمون
	\LTRfootnote{Herbert A. Simon}
	 که معتقدند تفکر انسان نیز در اصل مانند شیوه تفکر ماشین های محاسباتی بوده و هیچ مانعی برای پوشش تمامی فعالیت های ذهنی انسانی برای ماشین های محاسباتی وجود ندارد.
	\item   دسته دوم مانند کورت گودل 
	\LTRfootnote{Kurt Gödel}
	و جان سرل
	\LTRfootnote{John R Autor Searle}
	 که بعضی با تحقق پرسش اول و بعضی دیگر با تحقق پرسش دوم مخالفند.
	در دسته اول، کورت گودل با ارائه نظریه ناتمامیت خود امکان تحقق این امر را ماشین های محاسباتی رد می کند. در دسته دوم هم جان سرل با مطرح کردن آزمایش اتاق چینی امکان وجود ذهن و هوشیاری برای ماشین های محاسباتی را به چالش می کشد.
	\item  دسته سوم مانند تورینگ که با صرفنظر کردن از این پرسش، سعی کرده اند درباره این موضوع بحثی نکرده و روی رفتار و نتایج عمل های ماشین های محاسباتی تمرکز کنند. این دسته با داشتن یک توافق خوشبینانه، فرض می کنند که ماشین های محاسباتی نیز فکر می کنند.
\end{enumerate}


\section*{طرح پیشنهادی:}

\paragraph{} \hspace{0pt}
در این تحقیق با بررسی نظرات موافقین و مخالفین هر یک از دو پرسش اساسی که در مقدمه ذکر شد، امکان تحقق موضوع مطرح شده را بررسی می کنیم. ما در بررسی ادله موافقین و مخالفین ضمن صرفنظر کردن از وارد کردن و دخالت دادن دیدگاه های فلسفی و مذهبی، سعی در بررسی ادله موافقین و مخالفین از طریق جنبه های ریاضی و منطقی داریم.

\section*{محصولات طرح:}

\paragraph{} \hspace{0pt}
با توجه به نظری بودن بحث، محصولات انجام این پروژه شامل داده‌ها و مستنداتی خواهند بود که امکان داشتن یا نداشتن مسأله مطرح شده را توجیه خواهند کرد. همچنین این مستندات می توانند به پژوهش های بعدی جهت ارائه مدل بهتر و کاراتر برای پوشش سیستم ذهنی انسان کمک کنند.

\section*{مراحل انجام:}

\paragraph{} \hspace{0pt}
در این پروژه ابتدا مطالب مختصری از تست تورینگ( به همراه نقدهای وارد بر آن) و ساختار معماری فون نیومن( که اساس کار کامپیوتر های فعلی بر آن است) ارائه می دهیم. سپس وارد بحث درباره پاسخ دو پرسش ابتدایی خود خواهیم شد.
در وهله اول با بررسی سیستم تفکر انسان، به مقایسه آن با سیستم فکری ماشین های محاسباتی پرداخته و امکان حل مسائلی را که قابلیت حل شدن توسط انسان دارند؛ توسط ماشین های محاسباتی بررسی می کنیم.
در وهله بعدی ابتدا روی تعریف و نمود مشخصی از ذهن و هوشیاری در انسان متمرکز شویم. سپس با بررسی کردن ادله موافقین و مخالفین امکان پیاده سازی ذهن برای ماشین های محاسباتی، امکان تحقق این امر را هم بررسی می کنیم.

\section*{امکانات لازم:}

\paragraph{} \hspace{0pt}
با توجه به نظری بودن موضوع، امکانات قابل ذکری برای این انجام این تحقیق و پروژه لازم نیست.
\section*{زمان بندی:}

\paragraph{} \hspace{0pt}
با توجه به این که موضوع پروژه مطرح شده، موضوعی نظری بوده و هیچ گونه فعالیت عملی نیاز ندارد، با حذف مرحله «اجرای بخش عملی » زمانبندی خود را تدوین می کنیم. جداول زمانبندی مراحل مختلف به پیوست در صفحه بعد آورده شده اند.

\newpage

\begin{center}
\begin{tabular}{ |c|c|c|c|c|c|c|c|c|c|}
\hline
\multicolumn{10}{|c|}{جدول زمانبندی فعالیت شناسایی و تهیه منابع } \\
\hline
1&2&3&4&5&6&7&8&9&10\\
\hline
\cellcolor{black!100} & \cellcolor{black!100} & &  & &  &  & &  & \\
\hline
\cellcolor{black!100} & \cellcolor{black!100} & \cellcolor{black!100} & &  &  & &  & & \\
\hline
\cellcolor{black!100} & \cellcolor{black!100}& \cellcolor{black!100} &  & &  & & & & \\
\hline
\cellcolor{black!100} & \cellcolor{black!100} & \cellcolor{black!100} &  & &  & & & &  \\
\hline
\end{tabular}
\end{center}

\begin{center}
\begin{tabular}{ |c|c|c|c|c|c|c|c|c|c|}
\hline
\multicolumn{10}{|c|}{جدول زمانبندی فعالیت تنظیم ساختار } \\
\hline
1&2&3&4&5&6&7&8&9&10\\
\hline
 &  & & \cellcolor{black!100} & &  &  & &  & \\
\hline
 &  & & \cellcolor{black!100} & \cellcolor{black!100} &  & &  & & \\
\hline
 & &  & \cellcolor{black!100} & \cellcolor{black!100} & \cellcolor{black!100} & & & & \\
\hline
 &  & \cellcolor{black!100} & \cellcolor{black!100} & \cellcolor{black!100} & \cellcolor{black!100} & & & &  \\
\hline
\end{tabular}
\end{center}

\begin{center}
\begin{tabular}{ |c|c|c|c|c|c|c|c|c|c|}
\hline
\multicolumn{10}{|c|}{جدول زمانبندی فعالیت مطالعه و یادداشت برداری } \\
\hline
1&2&3&4&5&6&7&8&9&10\\
\hline
 &  & &  & & \cellcolor{black!100} &  & &  & \\
\hline
 &  & &  & \cellcolor{black!100} & \cellcolor{black!100}  &  &  & & \\
\hline
 & &  & \cellcolor{black!100} & \cellcolor{black!100} & \cellcolor{black!100} & \cellcolor{black!100} & & & \\
\hline
 &  &  & \cellcolor{black!100} & \cellcolor{black!100} & \cellcolor{black!100} & \cellcolor{black!100} & & &  \\
\hline
\end{tabular}
\end{center}

\begin{center}
\begin{tabular}{ |c|c|c|c|c|c|c|c|c|c|}
\hline
\multicolumn{10}{|c|}{جدول زمانبندی فعالیت تهیه پیش نویس اولیه } \\
\hline
1&2&3&4&5&6&7&8&9&10\\
\hline
 &  & &  & &  &  &\cellcolor{black!100} &  & \\
\hline
 &  & &  &  &  &  & \cellcolor{black!100} & & \\
\hline
 & &  &  &  &  & \cellcolor{black!100} & \cellcolor{black!100} & & \\
\hline
 &  &  &  &  &  & \cellcolor{black!100} & \cellcolor{black!100} & &  \\
\hline
\end{tabular}
\end{center}

\begin{center}
\begin{tabular}{ |c|c|c|c|c|c|c|c|c|c|}
\hline
\multicolumn{10}{|c|}{جدول زمانبندی فعالیت تهیه تهیه گزارش نهایی } \\
\hline
1&2&3&4&5&6&7&8&9&10\\
\hline
 &  & &  & &  &  & & \cellcolor{black!100} & \\
\hline
 &  & &  &  &  &  &  & \cellcolor{black!100}& \cellcolor{black!100} \\
\hline
 & &  &  &  &  &  &  & \cellcolor{black!100} & \cellcolor{black!100}  \\
\hline
 &  &  &  &  &  & &  & \cellcolor{black!100} & \cellcolor{black!100}  \\
\hline
\end{tabular}
\end{center}

\end{document}